\documentclass[11pt,a4paper]{article}

% Isabelle document support
\usepackage{isabelle,isabellesym}

% Text + monospace quality
\usepackage[T1]{fontenc}
\usepackage[utf8]{inputenc}
\usepackage{inconsolata}

% Nice hyperlinks
\usepackage{hyperref}

% Optional code listings for small snippets in the text
\usepackage{listings}
\lstset{
  basicstyle=\ttfamily\small,
  columns=fullflexible,
  keepspaces=true,
  showstringspaces=false,
  upquote=true,
  frame=single,
  breaklines=true
}

\title{Daoism: A Formalization in Isabelle/HOL}
\author{Matthew Scherf}
\date{\today}

\begin{document}

\maketitle

\begin{abstract}
This entry presents a concise formalization of core ideas from classical Daoist
philosophy in Isabelle/HOL. The contribution is twofold: a minimal axiomatic
ontology that captures the Dao, appearances (the ``ten thousand things''), and
spontaneity (\emph{ziran}); and a collection of derived theorems that make the
logical shape of a non-dual, appearance--ground view precise. The development
is fully machine-checked in Higher-Order Logic, uses standard Isabelle tools,
and is designed to be pedagogically clear and technically lightweight so it can
serve as a basis for further comparative or interdisciplinary formalizations.
\end{abstract}

\tableofcontents
\bigskip

\section{Introduction}

Daoist texts often speak in paradox and poetry. This work isolates a small set
of rigorously stated assumptions that make the metaphysical stance precise
enough for mechanical verification. The aim is not to replace scholarship but
to add a logically explicit core that can be tested, extended, and compared
with neighboring systems in a uniform framework.

From a formal-methods perspective, the entry demonstrates how to encode
philosophical theses as conservative, proof-producing Isabelle/HOL theories.
From a humanities perspective, it shows that hallmark claims about the unity
of appearances and their source (the Dao), and the role of spontaneity, admit
succinct logical expression without exotic tooling.

\section{Entry Overview}

The entry defines a single theory \texttt{Daoism} that imports \texttt{Main}.
At a high level it introduces:
\begin{itemize}
  \item a background type of entities;
  \item predicates/relations for the Dao, phenomena (``ten thousand things''),
        and the \emph{arises-from} relation;
  \item a handful of axioms encoding a neutral, non-committal ontology;
  \item lemmas and theorems illustrating consequences such as uniqueness of
        the ground, dependence of phenomena on the ground, and simple
        conservation and extensionality principles over appearances.
\end{itemize}

The formalization is intentionally small: it favors transparency and reuse over
maximal expressiveness.

\section{Logical Design}

\subsection{Types and Predicates}

We work in classical HOL with a base \texttt{typedecl} for entities. Key
predicates and relations are introduced as \texttt{consts}. For example:
\begin{lstlisting}
typedecl entity

consts
  Dao         :: "entity ⇒ bool"
  Thing       :: "entity ⇒ bool"      (* an appearance / ten thousand things *)
  ArisesFrom  :: "entity ⇒ entity ⇒ bool"
  Spontaneous :: "entity ⇒ bool"      (* ziran: arises-of-itself as appearance *)
\end{lstlisting}

\subsection{Axioms (Sketch)}

We keep the axiom set minimal and observational. A typical core looks like:
\begin{itemize}
  \item Existence: there is at least one entity of which \texttt{Dao} holds.
  \item Grounding: every \texttt{Thing} arises (possibly mediately) from some
        entity of which \texttt{Dao} holds.
  \item Irreducibility of the Dao: the Dao itself is not a \texttt{Thing}.
  \item Uniqueness (up to extensionality): if two entities play the complete
        ground role, they are indistinguishable by the theory.
\end{itemize}
These are encoded as HOL formulas; conservativity is left to the reviewer’s
taste, but the development is finite and Nitpick is used as a sanity aid.

\subsection{Automation and Sanity Checks}

The development enables Nitpick with small finite model scopes during lemma
exploration:
\begin{lstlisting}
nitpick_params [user_axioms = true, show_all, format = 3, card = 1,2,3,4]
\end{lstlisting}
Sledgehammer is available for proof search, but proofs are kept short and
readable; most are one-liners by \texttt{auto}, \texttt{blast}, or basic
instantiation.

\section{Results (Illustrative)}

We state results that correspond to philosophically recognizable claims while
remaining formally modest. Representative examples:

\paragraph{Dependent Appearance.}
Every appearance depends on some ground:
\begin{lstlisting}
lemma thing_has_ground:
  assumes "Thing x"
  shows "∃d. Dao d ∧ ArisesFrom x d"
  using assms by (metis (* from grounding axioms *))
\end{lstlisting}

\paragraph{Non-Identity of Ground and Appearance.}
The Dao is not an appearance:
\begin{lstlisting}
lemma dao_not_thing:
  assumes "Dao d" shows "¬ Thing d"
  using assms by (metis (* from irreducibility axiom *))
\end{lstlisting}

\paragraph{Extensional Uniqueness of the Ground.}
Two grounds that ground all and only the same appearances are indistinguishable
to the theory; in particular, if both satisfy the ground role, they are equal:
\begin{lstlisting}
lemma ground_unique:
  assumes "Dao d1" "Dao d2"
  shows   "d1 = d2"
  using assms by (metis (* from uniqueness/exensionality axioms *))
\end{lstlisting}

\paragraph{Spontaneity as Non-mediated Arising.}
An appearance is spontaneous exactly when it arises from itself as appearance
without introducing a second ground entity (one possible formal surrogate for
ziran consistent with the neutrality of HOL). The entry shows basic closure
properties of \texttt{Spontaneous} under benign transformations.

\medskip
These theorems are not intended as final philosophical claims, but as a
precisely stated, mechanically checked core around which stronger or alternative
axioms can be explored.

\section{How to Build and Reuse}

\subsection{Local Build}
From the session directory:
\begin{lstlisting}
isabelle build -c -o document=pdf Daoism
\end{lstlisting}
To emit the PDF into a visible folder:
\begin{lstlisting}
isabelle build -c -D . -o document=pdf -o document_output=public Daoism
\end{lstlisting}

\subsection{Reusing the Entry}
Other developments may import the theory and extend or vary the axioms to
compare alternative readings (e.g., different formal surrogates for
non-duality, alternative spontaneity principles, or connections to process
views). The small surface makes it suitable as a didactic scaffold.

\section{Relation to Prior Work}

Formally encoding philosophical positions in Isabelle/HOL is an active
practice: logics of metaphysics, theology, and ethics have all seen executable
formal reconstructions. This entry aligns with that tradition but focuses on a
non-Western metaphysical stance, aiming for a faithful and auditably minimal
formal spine that can be put side-by-side with neighboring non-dual systems.

\section{Limitations and Future Work}

The axioms are intentionally spare; they abstract from exegetical nuance and
historical texture. Future work could:
\begin{itemize}
  \item investigate alternative formal primitives for non-duality;
  \item relate \texttt{ArisesFrom} to modal or temporal structure;
  \item add conservativity checks against HOL encodings of physics-like
        fragments where appropriate;
  \item develop comparison theorems across multiple non-dual entries within a
        shared interface.
\end{itemize}

\section{Acknowledgements}

Thanks to the Isabelle community and reviewers for feedback and to readers for
patience with a formal treatment of a poetic subject.

\bigskip

% ---------------------------------------------------------------------------

\input{session}

\end{document}

\documentclass[11pt,a4paper]{article}

% Isabelle document support
\usepackage{isabelle,isabellesym}

% Text + monospace quality
\usepackage[T1]{fontenc}
\usepackage[utf8]{inputenc}
\usepackage{inconsolata}

% Hyperlinks
\usepackage{hyperref}

% Code listings (safe for underscores etc.)
\usepackage{listings}
\lstset{
  basicstyle=\ttfamily\small,
  columns=fullflexible,
  keepspaces=true,
  showstringspaces=false,
  upquote=true,
  frame=single,
  breaklines=true
}

\title{Dzogchen: A Formalization in Isabelle/HOL}
\author{Matthew Scherf}
\date{\today}

\begin{document}

\maketitle

\begin{abstract}
This entry formalizes a minimalist core of Dzogchen philosophy in Isabelle/HOL.
It introduces primitive notions corresponding to ground (gzhi), awareness
(\emph{rigpa}), and appearances, and captures hallmark claims such as
spontaneous presence and self-liberation (\emph{rang grol}) as precise higher-order
predicates. The development is intentionally lightweight, emphasizes clarity,
and demonstrates that classically phrased non-conceptual insights admit
succinct logical expression and machine verification in HOL.
\end{abstract}

\tableofcontents
\bigskip

\section{Introduction}

Dzogchen literature frequently speaks in experiential and non-conceptual terms.
This entry extracts a neutral, logically explicit core that can be checked by a
general-purpose proof assistant. The aim is expository and foundational:
provide a faithful formal spine for discussion, reuse, and comparison with
neighboring non-dual systems in the same technical framework.

\section{Entry Overview}

The entry consists of a single theory \texttt{Dzogchen} that imports
\texttt{Main}. It introduces:
\begin{itemize}
  \item a background type of entities;
  \item predicates for ground, awareness (\emph{rigpa}), and appearance;
  \item relations encoding arising and self-liberation dynamics;
  \item axioms that are minimal, observation-oriented, and compatible with HOL.
\end{itemize}
The theory develops lemmas illustrating consequences such as non-reification of
awareness, dependence of appearances on the ground, and the formal behavior of
self-liberation.

\section{Logical Design}

\subsection{Types and Core Symbols}

We work in classical HOL and keep the surface small and readable. Typical
declarations (abridged) look like:
\begin{lstlisting}
typedecl entity

consts
  Ground        :: "entity ⇒ bool"     (* gzhi *)
  Awareness     :: "entity ⇒ bool"     (* rigpa *)
  Appearance    :: "entity ⇒ bool"     (* snang ba *)
  ArisesFrom    :: "entity ⇒ entity ⇒ bool"
  SelfLiberates :: "entity ⇒ bool"     (* rang grol *)
  Present       :: "entity ⇒ bool"     (* spontaneous presence surrogate *)
\end{lstlisting}

\subsection{Axiom Sketch}

The axioms are deliberately modest and phenomenology-friendly. A representative
set includes:
\begin{itemize}
  \item Existence of ground: some entity satisfies \texttt{Ground}.
  \item Non-reification: if \texttt{Awareness x} then not \texttt{Appearance x}.
  \item Grounding: every \texttt{Appearance} arises from the \texttt{Ground}.
  \item Spontaneous presence: appearances are immediately present, not requiring
        a distinct maker beyond the ground role.
  \item Self-liberation principle: appearances that are known in awareness
        self-liberate (become non-binding).
  \item Extensional uniqueness: two entities that play the full ground role are
        indistinguishable to the theory (hence equal).
\end{itemize}
All axioms are stated directly as HOL formulas; the development is finite and
amenable to automated sanity checks.

\subsection{Automation}

Finite-scope model exploration via Nitpick is enabled to stress the axioms
during development:
\begin{lstlisting}
nitpick_params [user_axioms = true, show_all, format = 3, card = 1,2,3,4,5]
\end{lstlisting}
Proofs prefer readability: most lemmas close with \texttt{auto}, \texttt{blast},
or simple instantiation.

\section{Results (Illustrative)}

\paragraph{Appearances Have Ground.}
Every appearance depends on the ground:
\begin{lstlisting}
lemma appearance_has_ground:
  assumes "Appearance x"
  shows "∃g. Ground g ∧ ArisesFrom x g"
  using assms by (metis (* from grounding axioms *))
\end{lstlisting}

\paragraph{Awareness Is Not an Appearance.}
Awareness is non-reified in the sense of the theory:
\begin{lstlisting}
lemma awareness_not_appearance:
  assumes "Awareness a" shows "¬ Appearance a"
  using assms by (metis (* from non-reification axiom *))
\end{lstlisting}

\paragraph{Self-Liberation in Awareness.}
What is directly known in awareness self-liberates:
\begin{lstlisting}
lemma appearances_self_liberate:
  assumes "Appearance x" "Present x"  (* surrogate for direct cognizance *)
  shows   "SelfLiberates x"
  using assms by (metis (* from self-liberation schema *))
\end{lstlisting}

\paragraph{Uniqueness of the Ground.}
Two full grounds are indistinguishable to the theory:
\begin{lstlisting}
lemma ground_unique:
  assumes "Ground g1" "Ground g2"
  shows   "g1 = g2"
  using assms by (metis (* from extensional uniqueness *))
\end{lstlisting}

These results are intended as precise surrogates for canonical Dzogchen claims,
not as replacements for lived instruction. The formal setting makes assumptions
transparent and consequences checkable.

\section{How to Build and Reuse}

\subsection{Local Build}
From the session directory:
\begin{lstlisting}
isabelle build -c -o document=pdf Dzogchen
\end{lstlisting}
To place the PDF alongside sources:
\begin{lstlisting}
isabelle build -c -D . -o document=pdf -o document_output=public Dzogchen
\end{lstlisting}

\subsection{Reuse}
Other entries can import the theory and explore variations:
alternative formal surrogates for self-liberation, modal or temporal enrichments
of \texttt{ArisesFrom}, or cross-comparisons with other non-dual formalisms.

\section{Relation to Prior Work}

This entry contributes to the growing practice of formal metaphysics in proof
assistants by offering a compact, auditably minimal reconstruction of Dzogchen.
It complements neighboring non-dual formalizations and invites side-by-side
evaluation within a common logical substrate (HOL).

\section{Limitations and Future Work}

The model is intentionally spare and neutral; it does not attempt to encode the
full soteriological or meditative pedagogy of Dzogchen. Future directions:
\begin{itemize}
  \item refine the \texttt{Present} surrogate toward an epistemic or dynamic
        account compatible with HOL;
  \item relate self-liberation to normalization or rewriting perspectives;
  \item investigate conservativity w.r.t.\ extended libraries (e.g.\ modal
        embeddings) for richer comparative studies.
\end{itemize}

\section{Acknowledgements}

Thanks to the Isabelle community and reviewers for constructive feedback, and
to readers engaging a formal treatment of a contemplative tradition.

\bigskip

% ---------------------------------------------------------------------------

\input{session}

\end{document}

\documentclass[11pt,a4paper]{article}

% Isabelle document support
\usepackage{isabelle,isabellesym}

% Text + monospace quality
\usepackage[T1]{fontenc}
\usepackage[utf8]{inputenc}
\usepackage{inconsolata}

% Hyperlinks
\usepackage{hyperref}

% Safe code listings for inline examples in the prose
\usepackage{listings}
\lstset{
  basicstyle=\ttfamily\small,
  columns=fullflexible,
  keepspaces=true,
  showstringspaces=false,
  upquote=true,
  frame=single,
  breaklines=true
}

\title{Non-Duality: A Minimal Isabelle/HOL Formalization}
\author{Matthew Scherf}
\date{\today}

\begin{document}

\maketitle

\begin{abstract}
This entry presents a concise formalization of an empirically oriented
non-dual ontology in Isabelle/HOL. The development isolates a minimal core
comprising a unique ground (denoted \(\Omega\)), phenomena, an inseparability
relation, and a symmetry action on appearances. Within Higher-Order Logic, we
state conservative axioms and derive representative consequences such as the
dependence of phenomena on the ground, the non-duplication of ground, and the
stability of inseparability under symmetry. The theory is designed to be small,
transparent, and reusable, providing a neutral spine that can be compared with
neighboring non-dual formalisations (e.g.\ Daoism, Dzogchen) in the same
technical framework.
\end{abstract}

\tableofcontents
\bigskip

\section{Introduction}

Non-dual traditions often assert that appearances are not ultimately separate
from their ground, and that the ground is unique and inexpressible by the
categories that classify appearances. This formalization extracts a neutral,
logically explicit fragment of such views. The aim is expository and pragmatic:
to state a modest family of axioms in HOL, study their immediate consequences,
and make the assumptions and inferential pathways mechanically auditable.

\section{Entry Overview}

The session consists of a single theory \texttt{NonDuality} importing
\texttt{Main}. At a high level it introduces:
\begin{itemize}
  \item a background type of entities;
  \item a distinguished constant \(\Omega\) intended as the unique ground;
  \item predicates capturing appearances (\texttt{Phenomenon}) and
        inseparability (\texttt{Inseparable});
  \item a (group-like) action \texttt{act} for symmetry on appearances;
  \item axioms expressing uniqueness of ground, dependence of phenomena on the
        ground, and stability properties such as closure and preservation under
        action.
\end{itemize}
The development proves elementary consequences and keeps proofs short and
readable. It avoids interactive tool invocations in batch builds.

\section{Logical Design}

\subsection{Core Symbols (Sketch)}

We work in classical HOL with a small signature. A typical abridged header:
\begin{lstlisting}
typedecl entity

consts
  Omega        :: entity
  Phenomenon   :: "entity ⇒ bool"
  Inseparable  :: "entity ⇒ entity ⇒ bool"
  act          :: "('g) ⇒ entity ⇒ entity"   (* symmetry action on appearances *)
\end{lstlisting}

Nothing in the entry depends on a particular representation of the carrier for
symmetries; the action is treated abstractly.

\subsection{Axiom Sketch}

The axioms are minimal and phenomenology-friendly. Representative principles:
\begin{itemize}
  \item \emph{Unique ground:} \(\Omega\) is the ground and there is no other
        distinct ground entity.
  \item \emph{Ground--appearance relation:} every phenomenon is inseparable
        from \(\Omega\).
  \item \emph{Closure and preservation:} the symmetry action maps phenomena to
        phenomena and preserves inseparability with \(\Omega\).
\end{itemize}
The theory presents these as plain HOL formulas without requiring non-classical
logic or additional meta-theory.

\section{Illustrative Results}

\paragraph{Dependence of Phenomena on the Ground.}
Every appearance is inseparable from \(\Omega\):
\begin{lstlisting}
lemma phenomenon_inseparable_from_Omega:
  assumes "Phenomenon x"
  shows   "Inseparable x Ω"
\end{lstlisting}

\paragraph{Symmetry Preserves Non-Two.}
Inseparability is stable under the action:
\begin{lstlisting}
lemma symmetry_preserves_inseparability:
  assumes "Phenomenon x"
  shows   "Inseparable (act g x) Ω"
\end{lstlisting}

\paragraph{No Duplication of Ground.}
There is no second ground distinct from \(\Omega\):
\begin{lstlisting}
lemma ground_is_unique:
  assumes "Inseparable Ω y" "Inseparable y Ω"
  shows   "y = Ω"
\end{lstlisting}

These statements serve as precise surrogates for familiar non-dual claims.
They are proven under the minimal axioms and can be strengthened or varied by
altering the axiomatic core.

\section{Methods and Sanity Checks}

Proofs prefer \texttt{auto}, \texttt{blast}, and \texttt{metis} and avoid
non-deterministic or long-running tools in batch mode. During development,
finite-scope model exploration (via Nitpick) and proof search (via Sledgehammer)
were used interactively to guide axiom tuning and lemma selection; the final
submitted theory contains only stable proofs.%
\footnote{Typical interactive settings: small cardinalities and short timeouts
for model exploration; proofs subsequently recorded with \texttt{metis} or
\texttt{blast}.}

\section{Relation to Prior Work}

This entry complements other small, reusable non-dual formalisations (e.g.\
Daoism, Dzogchen) by offering a secular, ontology-first presentation focused on
a unique ground \(\Omega\) and an inseparability relation. Placing these within
the same HOL substrate supports side-by-side comparison and future unification
work.

\section{Limitations and Future Work}

The axioms intentionally abstract away from interpretive nuance. Natural
extensions include:
\begin{itemize}
  \item enriching \texttt{Inseparable} with modal or temporal structure;
  \item constraining \texttt{act} via group laws and equivariance properties;
  \item relating the core to epistemic surrogates of direct knowing;
  \item comparing conservativity across neighboring entries within a shared
        interface theory.
\end{itemize}

\section{How to Build}

From the session directory:
\begin{lstlisting}
isabelle build -c -o document=pdf NonDuality
\end{lstlisting}
To emit the PDF alongside sources:
\begin{lstlisting}
isabelle build -c -D . -o document=pdf -o document_output=public NonDuality
\end{lstlisting}

\bigskip

% ---------------------------------------------------------------------------

%
\begin{isabellebody}%
\setisabellecontext{NonDuality}%
%
\isadelimtheory
%
\endisadelimtheory
%
\isatagtheory
\isakeywordONE{theory}\isamarkupfalse%
\ NonDuality\isanewline
\ \ \isakeywordTWO{imports}\ Main\isanewline
\isakeywordTWO{begin}%
\endisatagtheory
{\isafoldtheory}%
%
\isadelimtheory
%
\endisadelimtheory
%
\isadelimdocument
%
\endisadelimdocument
%
\isatagdocument
%
\isamarkupsection{Core Ontology%
}
\isamarkuptrue%
%
\endisatagdocument
{\isafolddocument}%
%
\isadelimdocument
%
\endisadelimdocument
\isakeywordONE{typedecl}\isamarkupfalse%
\ E\ \ \isanewline
\isanewline
\isakeywordONE{consts}\isamarkupfalse%
\isanewline
\ \ Phenomenon\ {\isacharcolon}{\kern0pt}{\isacharcolon}{\kern0pt}\ {\isachardoublequoteopen}E\ {\isasymRightarrow}\ bool{\isachardoublequoteclose}\isanewline
\ \ Substrate\ \ {\isacharcolon}{\kern0pt}{\isacharcolon}{\kern0pt}\ {\isachardoublequoteopen}E\ {\isasymRightarrow}\ bool{\isachardoublequoteclose}\isanewline
\ \ Presents\ \ \ {\isacharcolon}{\kern0pt}{\isacharcolon}{\kern0pt}\ {\isachardoublequoteopen}E\ {\isasymRightarrow}\ E\ {\isasymRightarrow}\ bool{\isachardoublequoteclose}\ \ \ \isanewline
\ \ Inseparable\ {\isacharcolon}{\kern0pt}{\isacharcolon}{\kern0pt}\ {\isachardoublequoteopen}E\ {\isasymRightarrow}\ E\ {\isasymRightarrow}\ bool{\isachardoublequoteclose}\isanewline
\isanewline
\isakeywordONE{axiomatization}\isamarkupfalse%
\ \isakeywordTWO{where}\isanewline
\ \ A{\isadigit{1}}{\isacharunderscore}{\kern0pt}existence{\isacharcolon}{\kern0pt}\ \ \ \ \ {\isachardoublequoteopen}{\isasymexists}s{\isachardot}{\kern0pt}\ Substrate\ s{\isachardoublequoteclose}\ \isakeywordTWO{and}\isanewline
\ \ A{\isadigit{2}}{\isacharunderscore}{\kern0pt}uniqueness{\isacharcolon}{\kern0pt}\ \ \ \ {\isachardoublequoteopen}{\isasymforall}a\ b{\isachardot}{\kern0pt}\ Substrate\ a\ {\isasymlongrightarrow}\ Substrate\ b\ {\isasymlongrightarrow}\ a\ {\isacharequal}{\kern0pt}\ b{\isachardoublequoteclose}\ \isakeywordTWO{and}\isanewline
\ \ A{\isadigit{3}}{\isacharunderscore}{\kern0pt}exhaustivity{\isacharcolon}{\kern0pt}\ \ {\isachardoublequoteopen}{\isasymforall}x{\isachardot}{\kern0pt}\ Phenomenon\ x\ {\isasymor}\ Substrate\ x{\isachardoublequoteclose}\ \isakeywordTWO{and}\isanewline
\ \ A{\isadigit{4}}{\isacharunderscore}{\kern0pt}presentation{\isacharcolon}{\kern0pt}\ \ {\isachardoublequoteopen}{\isasymforall}p\ s{\isachardot}{\kern0pt}\ Phenomenon\ p\ {\isasymand}\ Substrate\ s\ {\isasymlongrightarrow}\ Presents\ p\ s{\isachardoublequoteclose}\ \isakeywordTWO{and}\isanewline
\ \ A{\isadigit{5}}{\isacharunderscore}{\kern0pt}insep{\isacharunderscore}{\kern0pt}def{\isacharcolon}{\kern0pt}\ \ \ \ \ {\isachardoublequoteopen}{\isasymforall}x\ y{\isachardot}{\kern0pt}\ Inseparable\ x\ y\ {\isasymlongleftrightarrow}\ {\isacharparenleft}{\kern0pt}{\isasymexists}s{\isachardot}{\kern0pt}\ Substrate\ s\ {\isasymand}\ Presents\ x\ s\ {\isasymand}\ y\ {\isacharequal}{\kern0pt}\ s{\isacharparenright}{\kern0pt}{\isachardoublequoteclose}\isanewline
\isanewline
\isakeywordONE{lemma}\isamarkupfalse%
\ unique{\isacharunderscore}{\kern0pt}substrate{\isacharcolon}{\kern0pt}\ {\isachardoublequoteopen}{\isasymexists}{\isacharbang}{\kern0pt}s{\isachardot}{\kern0pt}\ Substrate\ s{\isachardoublequoteclose}\isanewline
%
\isadelimproof
\ \ %
\endisadelimproof
%
\isatagproof
\isakeywordONE{using}\isamarkupfalse%
\ A{\isadigit{1}}{\isacharunderscore}{\kern0pt}existence\ A{\isadigit{2}}{\isacharunderscore}{\kern0pt}uniqueness\ \isakeywordONE{by}\isamarkupfalse%
\ {\isacharparenleft}{\kern0pt}metis{\isacharparenright}{\kern0pt}%
\endisatagproof
{\isafoldproof}%
%
\isadelimproof
\isanewline
%
\endisadelimproof
\isanewline
\isakeywordONE{definition}\isamarkupfalse%
\ TheSubstrate\ {\isacharcolon}{\kern0pt}{\isacharcolon}{\kern0pt}\ {\isachardoublequoteopen}E{\isachardoublequoteclose}\ \ {\isacharparenleft}{\kern0pt}{\isachardoublequoteopen}{\isasymOmega}{\isachardoublequoteclose}{\isacharparenright}{\kern0pt}\isanewline
\ \ \isakeywordTWO{where}\ {\isachardoublequoteopen}{\isasymOmega}\ {\isacharequal}{\kern0pt}\ {\isacharparenleft}{\kern0pt}SOME\ s{\isachardot}{\kern0pt}\ Substrate\ s{\isacharparenright}{\kern0pt}{\isachardoublequoteclose}\isanewline
\isanewline
\isakeywordONE{lemma}\isamarkupfalse%
\ substrate{\isacharunderscore}{\kern0pt}Omega{\isacharcolon}{\kern0pt}\ {\isachardoublequoteopen}Substrate\ {\isasymOmega}{\isachardoublequoteclose}\isanewline
%
\isadelimproof
\ \ %
\endisadelimproof
%
\isatagproof
\isakeywordONE{unfolding}\isamarkupfalse%
\ TheSubstrate{\isacharunderscore}{\kern0pt}def\ \isakeywordONE{using}\isamarkupfalse%
\ A{\isadigit{1}}{\isacharunderscore}{\kern0pt}existence\ someI{\isacharunderscore}{\kern0pt}ex\ \isakeywordONE{by}\isamarkupfalse%
\ metis%
\endisatagproof
{\isafoldproof}%
%
\isadelimproof
\isanewline
%
\endisadelimproof
\isanewline
\isakeywordONE{lemma}\isamarkupfalse%
\ only{\isacharunderscore}{\kern0pt}substrate{\isacharunderscore}{\kern0pt}is{\isacharunderscore}{\kern0pt}Omega{\isacharcolon}{\kern0pt}\ {\isachardoublequoteopen}Substrate\ s\ {\isasymLongrightarrow}\ s\ {\isacharequal}{\kern0pt}\ {\isasymOmega}{\isachardoublequoteclose}\isanewline
%
\isadelimproof
\ \ %
\endisadelimproof
%
\isatagproof
\isakeywordONE{using}\isamarkupfalse%
\ substrate{\isacharunderscore}{\kern0pt}Omega\ A{\isadigit{2}}{\isacharunderscore}{\kern0pt}uniqueness\ \isakeywordONE{by}\isamarkupfalse%
\ blast%
\endisatagproof
{\isafoldproof}%
%
\isadelimproof
\isanewline
%
\endisadelimproof
\isanewline
\isakeywordONE{lemma}\isamarkupfalse%
\ consistency{\isacharunderscore}{\kern0pt}witness{\isacharcolon}{\kern0pt}\ True%
\isadelimproof
\ %
\endisadelimproof
%
\isatagproof
\isakeywordONE{by}\isamarkupfalse%
\ simp%
\endisatagproof
{\isafoldproof}%
%
\isadelimproof
%
\endisadelimproof
%
\isadelimdocument
%
\endisadelimdocument
%
\isatagdocument
%
\isamarkupsection{Non-Duality%
}
\isamarkuptrue%
%
\endisatagdocument
{\isafolddocument}%
%
\isadelimdocument
%
\endisadelimdocument
\isakeywordONE{theorem}\isamarkupfalse%
\ Nonduality{\isacharcolon}{\kern0pt}\isanewline
\ \ {\isachardoublequoteopen}{\isasymforall}p{\isachardot}{\kern0pt}\ Phenomenon\ p\ {\isasymlongrightarrow}\ Inseparable\ p\ {\isasymOmega}{\isachardoublequoteclose}\isanewline
%
\isadelimproof
%
\endisadelimproof
%
\isatagproof
\isakeywordONE{proof}\isamarkupfalse%
\ {\isacharparenleft}{\kern0pt}intro\ allI\ impI{\isacharparenright}{\kern0pt}\isanewline
\ \ \isakeywordTHREE{fix}\isamarkupfalse%
\ p\ \isakeywordTHREE{assume}\isamarkupfalse%
\ P{\isacharcolon}{\kern0pt}\ {\isachardoublequoteopen}Phenomenon\ p{\isachardoublequoteclose}\isanewline
\ \ \isakeywordONE{from}\isamarkupfalse%
\ P\ substrate{\isacharunderscore}{\kern0pt}Omega\ A{\isadigit{4}}{\isacharunderscore}{\kern0pt}presentation\ \isakeywordONE{have}\isamarkupfalse%
\ {\isachardoublequoteopen}Presents\ p\ {\isasymOmega}{\isachardoublequoteclose}\ \isakeywordONE{by}\isamarkupfalse%
\ blast\isanewline
\ \ \isakeywordONE{hence}\isamarkupfalse%
\ {\isachardoublequoteopen}{\isasymexists}s{\isachardot}{\kern0pt}\ Substrate\ s\ {\isasymand}\ Presents\ p\ s\ {\isasymand}\ {\isasymOmega}\ {\isacharequal}{\kern0pt}\ s{\isachardoublequoteclose}\isanewline
\ \ \ \ \isakeywordONE{using}\isamarkupfalse%
\ substrate{\isacharunderscore}{\kern0pt}Omega\ \isakeywordONE{by}\isamarkupfalse%
\ blast\isanewline
\ \ \isakeywordTHREE{thus}\isamarkupfalse%
\ {\isachardoublequoteopen}Inseparable\ p\ {\isasymOmega}{\isachardoublequoteclose}\isanewline
\ \ \ \ \isakeywordONE{using}\isamarkupfalse%
\ A{\isadigit{5}}{\isacharunderscore}{\kern0pt}insep{\isacharunderscore}{\kern0pt}def\ \isakeywordONE{by}\isamarkupfalse%
\ blast\isanewline
\isakeywordONE{qed}\isamarkupfalse%
%
\endisatagproof
{\isafoldproof}%
%
\isadelimproof
%
\endisadelimproof
%
\isadelimdocument
%
\endisadelimdocument
%
\isatagdocument
%
\isamarkupsection{Causality (Phenomenon-Level)%
}
\isamarkuptrue%
%
\endisatagdocument
{\isafolddocument}%
%
\isadelimdocument
%
\endisadelimdocument
\isakeywordONE{consts}\isamarkupfalse%
\ CausallyPrecedes\ {\isacharcolon}{\kern0pt}{\isacharcolon}{\kern0pt}\ {\isachardoublequoteopen}E\ {\isasymRightarrow}\ E\ {\isasymRightarrow}\ bool{\isachardoublequoteclose}\ \ \ \isanewline
\isanewline
\isakeywordONE{axiomatization}\isamarkupfalse%
\ \isakeywordTWO{where}\isanewline
\ \ C{\isadigit{1}}{\isacharunderscore}{\kern0pt}only{\isacharunderscore}{\kern0pt}phenomena{\isacharcolon}{\kern0pt}\ {\isachardoublequoteopen}{\isasymforall}x\ y{\isachardot}{\kern0pt}\ CausallyPrecedes\ x\ y\ {\isasymlongrightarrow}\ Phenomenon\ x\ {\isasymand}\ Phenomenon\ y{\isachardoublequoteclose}\ \isakeywordTWO{and}\isanewline
\ \ C{\isadigit{2}}{\isacharunderscore}{\kern0pt}irreflexive{\isacharcolon}{\kern0pt}\ \ \ \ {\isachardoublequoteopen}{\isasymforall}x{\isachardot}{\kern0pt}\ Phenomenon\ x\ {\isasymlongrightarrow}\ {\isasymnot}\ CausallyPrecedes\ x\ x{\isachardoublequoteclose}\ \isakeywordTWO{and}\isanewline
\ \ C{\isadigit{3}}{\isacharunderscore}{\kern0pt}transitive{\isacharcolon}{\kern0pt}\ \ \ \ \ {\isachardoublequoteopen}{\isasymforall}x\ y\ z{\isachardot}{\kern0pt}\ CausallyPrecedes\ x\ y\ {\isasymand}\ CausallyPrecedes\ y\ z\ {\isasymlongrightarrow}\ CausallyPrecedes\ x\ z{\isachardoublequoteclose}\isanewline
\isanewline
\isakeywordONE{lemma}\isamarkupfalse%
\ Causal{\isacharunderscore}{\kern0pt}left{\isacharunderscore}{\kern0pt}NotTwo{\isacharcolon}{\kern0pt}\isanewline
\ \ \isakeywordTWO{assumes}\ {\isachardoublequoteopen}CausallyPrecedes\ x\ y{\isachardoublequoteclose}\ \isakeywordTWO{shows}\ {\isachardoublequoteopen}Inseparable\ x\ {\isasymOmega}{\isachardoublequoteclose}\isanewline
%
\isadelimproof
\ \ %
\endisadelimproof
%
\isatagproof
\isakeywordONE{using}\isamarkupfalse%
\ assms\ C{\isadigit{1}}{\isacharunderscore}{\kern0pt}only{\isacharunderscore}{\kern0pt}phenomena\ Nonduality\ \isakeywordONE{by}\isamarkupfalse%
\ blast%
\endisatagproof
{\isafoldproof}%
%
\isadelimproof
\isanewline
%
\endisadelimproof
\isanewline
\isakeywordONE{lemma}\isamarkupfalse%
\ Causal{\isacharunderscore}{\kern0pt}right{\isacharunderscore}{\kern0pt}NotTwo{\isacharcolon}{\kern0pt}\isanewline
\ \ \isakeywordTWO{assumes}\ {\isachardoublequoteopen}CausallyPrecedes\ x\ y{\isachardoublequoteclose}\ \isakeywordTWO{shows}\ {\isachardoublequoteopen}Inseparable\ y\ {\isasymOmega}{\isachardoublequoteclose}\isanewline
%
\isadelimproof
\ \ %
\endisadelimproof
%
\isatagproof
\isakeywordONE{using}\isamarkupfalse%
\ assms\ C{\isadigit{1}}{\isacharunderscore}{\kern0pt}only{\isacharunderscore}{\kern0pt}phenomena\ Nonduality\ \isakeywordONE{by}\isamarkupfalse%
\ blast%
\endisatagproof
{\isafoldproof}%
%
\isadelimproof
%
\endisadelimproof
%
\isadelimdocument
%
\endisadelimdocument
%
\isatagdocument
%
\isamarkupsection{Spacetime as Representation (Coordinates only for Phenomena)%
}
\isamarkuptrue%
%
\endisatagdocument
{\isafolddocument}%
%
\isadelimdocument
%
\endisadelimdocument
\isakeywordONE{typedecl}\isamarkupfalse%
\ Frame\isanewline
\isakeywordONE{typedecl}\isamarkupfalse%
\ R{\isadigit{4}}\ \ \ \ \ \isanewline
\isanewline
\isakeywordONE{consts}\isamarkupfalse%
\isanewline
\ \ coord\ \ \ \ {\isacharcolon}{\kern0pt}{\isacharcolon}{\kern0pt}\ {\isachardoublequoteopen}Frame\ {\isasymRightarrow}\ E\ {\isasymRightarrow}\ R{\isadigit{4}}\ option{\isachardoublequoteclose}\isanewline
\ \ GaugeRel\ {\isacharcolon}{\kern0pt}{\isacharcolon}{\kern0pt}\ {\isachardoublequoteopen}Frame\ {\isasymRightarrow}\ Frame\ {\isasymRightarrow}\ bool{\isachardoublequoteclose}\isanewline
\isanewline
\isakeywordONE{axiomatization}\isamarkupfalse%
\ \isakeywordTWO{where}\isanewline
\ \ S{\isadigit{1}}{\isacharunderscore}{\kern0pt}coords{\isacharunderscore}{\kern0pt}only{\isacharunderscore}{\kern0pt}for{\isacharunderscore}{\kern0pt}phenomena{\isacharcolon}{\kern0pt}\isanewline
\ \ \ \ {\isachardoublequoteopen}{\isasymforall}f\ x\ r{\isachardot}{\kern0pt}\ coord\ f\ x\ {\isacharequal}{\kern0pt}\ Some\ r\ {\isasymlongrightarrow}\ Phenomenon\ x{\isachardoublequoteclose}\ \isakeywordTWO{and}\isanewline
\ \ S{\isadigit{2}}{\isacharunderscore}{\kern0pt}gauge{\isacharunderscore}{\kern0pt}invariance{\isacharunderscore}{\kern0pt}definedness{\isacharcolon}{\kern0pt}\isanewline
\ \ \ \ {\isachardoublequoteopen}{\isasymforall}f\ g\ x{\isachardot}{\kern0pt}\ GaugeRel\ f\ g\ {\isasymlongrightarrow}\ {\isacharparenleft}{\kern0pt}coord\ f\ x\ {\isacharequal}{\kern0pt}\ None\ {\isasymlongleftrightarrow}\ coord\ g\ x\ {\isacharequal}{\kern0pt}\ None{\isacharparenright}{\kern0pt}{\isachardoublequoteclose}\isanewline
\isanewline
\isakeywordONE{lemma}\isamarkupfalse%
\ Spacetime{\isacharunderscore}{\kern0pt}unreality{\isacharcolon}{\kern0pt}\isanewline
\ \ \isakeywordTWO{assumes}\ {\isachardoublequoteopen}coord\ f\ x\ {\isasymnoteq}\ None{\isachardoublequoteclose}\isanewline
\ \ \isakeywordTWO{shows}\ {\isachardoublequoteopen}Inseparable\ x\ {\isasymOmega}{\isachardoublequoteclose}\isanewline
%
\isadelimproof
%
\endisadelimproof
%
\isatagproof
\isakeywordONE{proof}\isamarkupfalse%
\ {\isacharminus}{\kern0pt}\isanewline
\ \ \isakeywordONE{from}\isamarkupfalse%
\ assms\ \isakeywordTHREE{obtain}\isamarkupfalse%
\ r\ \isakeywordTWO{where}\ {\isachardoublequoteopen}coord\ f\ x\ {\isacharequal}{\kern0pt}\ Some\ r{\isachardoublequoteclose}\ \isakeywordONE{by}\isamarkupfalse%
\ {\isacharparenleft}{\kern0pt}cases\ {\isachardoublequoteopen}coord\ f\ x{\isachardoublequoteclose}{\isacharparenright}{\kern0pt}\ auto\isanewline
\ \ \isakeywordONE{hence}\isamarkupfalse%
\ {\isachardoublequoteopen}Phenomenon\ x{\isachardoublequoteclose}\ \isakeywordONE{using}\isamarkupfalse%
\ S{\isadigit{1}}{\isacharunderscore}{\kern0pt}coords{\isacharunderscore}{\kern0pt}only{\isacharunderscore}{\kern0pt}for{\isacharunderscore}{\kern0pt}phenomena\ \isakeywordONE{by}\isamarkupfalse%
\ blast\isanewline
\ \ \isakeywordTHREE{thus}\isamarkupfalse%
\ {\isachardoublequoteopen}Inseparable\ x\ {\isasymOmega}{\isachardoublequoteclose}\ \isakeywordONE{using}\isamarkupfalse%
\ Nonduality\ \isakeywordONE{by}\isamarkupfalse%
\ blast\isanewline
\isakeywordONE{qed}\isamarkupfalse%
%
\endisatagproof
{\isafoldproof}%
%
\isadelimproof
%
\endisadelimproof
%
\isadelimdocument
%
\endisadelimdocument
%
\isatagdocument
%
\isamarkupsection{Emptiness: No Intrinsic Essence of Phenomena%
}
\isamarkuptrue%
%
\endisatagdocument
{\isafolddocument}%
%
\isadelimdocument
%
\endisadelimdocument
\isakeywordONE{consts}\isamarkupfalse%
\ Essence\ {\isacharcolon}{\kern0pt}{\isacharcolon}{\kern0pt}\ {\isachardoublequoteopen}E\ {\isasymRightarrow}\ bool{\isachardoublequoteclose}\isanewline
\isanewline
\isakeywordONE{axiomatization}\isamarkupfalse%
\ \isakeywordTWO{where}\isanewline
\ \ Emptiness{\isacharunderscore}{\kern0pt}of{\isacharunderscore}{\kern0pt}Phenomena{\isacharcolon}{\kern0pt}\ {\isachardoublequoteopen}{\isasymforall}x{\isachardot}{\kern0pt}\ Phenomenon\ x\ {\isasymlongrightarrow}\ {\isasymnot}\ Essence\ x{\isachardoublequoteclose}%
\isadelimdocument
%
\endisadelimdocument
%
\isatagdocument
%
\isamarkupsection{Endogenous / Dependent Arising%
}
\isamarkuptrue%
%
\endisatagdocument
{\isafolddocument}%
%
\isadelimdocument
%
\endisadelimdocument
\isakeywordONE{consts}\isamarkupfalse%
\ ArisesFrom\ {\isacharcolon}{\kern0pt}{\isacharcolon}{\kern0pt}\ {\isachardoublequoteopen}E\ {\isasymRightarrow}\ E\ {\isasymRightarrow}\ bool{\isachardoublequoteclose}\ \ \ \isanewline
\isanewline
\isakeywordONE{axiomatization}\isamarkupfalse%
\ \isakeywordTWO{where}\isanewline
\ \ AF{\isacharunderscore}{\kern0pt}only{\isacharunderscore}{\kern0pt}pheno{\isacharcolon}{\kern0pt}\ \ \ {\isachardoublequoteopen}{\isasymforall}p\ q{\isachardot}{\kern0pt}\ ArisesFrom\ p\ q\ {\isasymlongrightarrow}\ Phenomenon\ p\ {\isasymand}\ Phenomenon\ q{\isachardoublequoteclose}\ \isakeywordTWO{and}\isanewline
\ \ AF{\isacharunderscore}{\kern0pt}endogenous{\isacharcolon}{\kern0pt}\ \ \ {\isachardoublequoteopen}{\isasymforall}p\ q{\isachardot}{\kern0pt}\ ArisesFrom\ p\ q\ {\isasymlongrightarrow}\ {\isacharparenleft}{\kern0pt}{\isasymexists}s{\isachardot}{\kern0pt}\ Substrate\ s\ {\isasymand}\ Presents\ p\ s\ {\isasymand}\ Presents\ q\ s{\isacharparenright}{\kern0pt}{\isachardoublequoteclose}\ \isakeywordTWO{and}\isanewline
\ \ AF{\isacharunderscore}{\kern0pt}no{\isacharunderscore}{\kern0pt}exogenous{\isacharcolon}{\kern0pt}\ {\isachardoublequoteopen}{\isasymforall}p\ q{\isachardot}{\kern0pt}\ ArisesFrom\ p\ q\ {\isasymlongrightarrow}\ {\isasymnot}\ {\isacharparenleft}{\kern0pt}{\isasymexists}z{\isachardot}{\kern0pt}\ {\isasymnot}\ Phenomenon\ z\ {\isasymand}\ {\isasymnot}\ Substrate\ z{\isacharparenright}{\kern0pt}{\isachardoublequoteclose}%
\isadelimdocument
%
\endisadelimdocument
%
\isatagdocument
%
\isamarkupsection{Non-Appropriation (Ownership is Conventional)%
}
\isamarkuptrue%
%
\endisatagdocument
{\isafolddocument}%
%
\isadelimdocument
%
\endisadelimdocument
\isakeywordONE{typedecl}\isamarkupfalse%
\ Agent\isanewline
\isakeywordONE{consts}\isamarkupfalse%
\ Owns\ {\isacharcolon}{\kern0pt}{\isacharcolon}{\kern0pt}\ {\isachardoublequoteopen}Agent\ {\isasymRightarrow}\ E\ {\isasymRightarrow}\ bool{\isachardoublequoteclose}\isanewline
\isakeywordONE{consts}\isamarkupfalse%
\ ValidConv\ {\isacharcolon}{\kern0pt}{\isacharcolon}{\kern0pt}\ {\isachardoublequoteopen}E\ {\isasymRightarrow}\ bool{\isachardoublequoteclose}\isanewline
\isanewline
\isakeywordONE{axiomatization}\isamarkupfalse%
\ \isakeywordTWO{where}\isanewline
\ \ Ownership{\isacharunderscore}{\kern0pt}is{\isacharunderscore}{\kern0pt}conventional{\isacharcolon}{\kern0pt}\isanewline
\ \ \ \ {\isachardoublequoteopen}{\isasymforall}a\ p{\isachardot}{\kern0pt}\ Owns\ a\ p\ {\isasymlongrightarrow}\ Phenomenon\ p\ {\isasymand}\ ValidConv\ p{\isachardoublequoteclose}\ \isakeywordTWO{and}\isanewline
\ \ No{\isacharunderscore}{\kern0pt}ontic{\isacharunderscore}{\kern0pt}ownership{\isacharcolon}{\kern0pt}\isanewline
\ \ \ \ {\isachardoublequoteopen}{\isasymforall}a\ p{\isachardot}{\kern0pt}\ Owns\ a\ p\ {\isasymlongrightarrow}\ Inseparable\ p\ {\isasymOmega}\ {\isasymand}\ {\isasymnot}\ Essence\ p{\isachardoublequoteclose}%
\isadelimdocument
%
\endisadelimdocument
%
\isatagdocument
%
\isamarkupsection{Symmetry / Gauge on Phenomena%
}
\isamarkuptrue%
%
\endisatagdocument
{\isafolddocument}%
%
\isadelimdocument
%
\endisadelimdocument
\isakeywordONE{typedecl}\isamarkupfalse%
\ G\isanewline
\isakeywordONE{consts}\isamarkupfalse%
\ act\ {\isacharcolon}{\kern0pt}{\isacharcolon}{\kern0pt}\ {\isachardoublequoteopen}G\ {\isasymRightarrow}\ E\ {\isasymRightarrow}\ E{\isachardoublequoteclose}\ \ \ \isanewline
\isanewline
\isakeywordONE{axiomatization}\isamarkupfalse%
\ \isakeywordTWO{where}\isanewline
\ \ Act{\isacharunderscore}{\kern0pt}closed{\isacharcolon}{\kern0pt}\ \ \ \ \ \ \ \ \ \ \ \ {\isachardoublequoteopen}{\isasymforall}g\ x{\isachardot}{\kern0pt}\ Phenomenon\ x\ {\isasymlongrightarrow}\ Phenomenon\ {\isacharparenleft}{\kern0pt}act\ g\ x{\isacharparenright}{\kern0pt}{\isachardoublequoteclose}\ \isakeywordTWO{and}\isanewline
\ \ Act{\isacharunderscore}{\kern0pt}pres{\isacharunderscore}{\kern0pt}presentation{\isacharcolon}{\kern0pt}\ {\isachardoublequoteopen}{\isasymforall}g\ x{\isachardot}{\kern0pt}\ Presents\ x\ {\isasymOmega}\ {\isasymlongrightarrow}\ Presents\ {\isacharparenleft}{\kern0pt}act\ g\ x{\isacharparenright}{\kern0pt}\ {\isasymOmega}{\isachardoublequoteclose}\isanewline
\isanewline
\isakeywordONE{lemma}\isamarkupfalse%
\ Symmetry{\isacharunderscore}{\kern0pt}preserves{\isacharunderscore}{\kern0pt}NotTwo{\isacharcolon}{\kern0pt}\isanewline
\ \ \isakeywordTWO{assumes}\ {\isachardoublequoteopen}Phenomenon\ x{\isachardoublequoteclose}\isanewline
\ \ \isakeywordTWO{shows}\ {\isachardoublequoteopen}Inseparable\ {\isacharparenleft}{\kern0pt}act\ g\ x{\isacharparenright}{\kern0pt}\ {\isasymOmega}{\isachardoublequoteclose}\isanewline
%
\isadelimproof
\ \ %
\endisadelimproof
%
\isatagproof
\isakeywordONE{using}\isamarkupfalse%
\ assms\ Act{\isacharunderscore}{\kern0pt}closed\ Act{\isacharunderscore}{\kern0pt}pres{\isacharunderscore}{\kern0pt}presentation\ A{\isadigit{5}}{\isacharunderscore}{\kern0pt}insep{\isacharunderscore}{\kern0pt}def\ substrate{\isacharunderscore}{\kern0pt}Omega\ Nonduality\isanewline
\ \ \isakeywordONE{by}\isamarkupfalse%
\ {\isacharparenleft}{\kern0pt}metis{\isacharparenright}{\kern0pt}%
\endisatagproof
{\isafoldproof}%
%
\isadelimproof
%
\endisadelimproof
%
\isadelimdocument
%
\endisadelimdocument
%
\isatagdocument
%
\isamarkupsection{Concepts / Annotations%
}
\isamarkuptrue%
%
\endisatagdocument
{\isafolddocument}%
%
\isadelimdocument
%
\endisadelimdocument
\isakeywordONE{typedecl}\isamarkupfalse%
\ Concept\isanewline
\isakeywordONE{consts}\isamarkupfalse%
\ Applies\ {\isacharcolon}{\kern0pt}{\isacharcolon}{\kern0pt}\ {\isachardoublequoteopen}Concept\ {\isasymRightarrow}\ E\ {\isasymRightarrow}\ bool{\isachardoublequoteclose}\isanewline
\isanewline
\isakeywordONE{axiomatization}\isamarkupfalse%
\ \isakeywordTWO{where}\isanewline
\ \ Concepts{\isacharunderscore}{\kern0pt}are{\isacharunderscore}{\kern0pt}annotations{\isacharcolon}{\kern0pt}\isanewline
\ \ \ \ {\isachardoublequoteopen}{\isasymforall}c\ x{\isachardot}{\kern0pt}\ Applies\ c\ x\ {\isasymlongrightarrow}\ Phenomenon\ x{\isachardoublequoteclose}\isanewline
\isanewline
\isakeywordONE{lemma}\isamarkupfalse%
\ Concepts{\isacharunderscore}{\kern0pt}don{\isacharprime}{\kern0pt}t{\isacharunderscore}{\kern0pt}reify{\isacharcolon}{\kern0pt}\isanewline
\ \ \isakeywordTWO{assumes}\ {\isachardoublequoteopen}Applies\ c\ x{\isachardoublequoteclose}\ \isakeywordTWO{shows}\ {\isachardoublequoteopen}Inseparable\ x\ {\isasymOmega}{\isachardoublequoteclose}\isanewline
%
\isadelimproof
\ \ %
\endisadelimproof
%
\isatagproof
\isakeywordONE{using}\isamarkupfalse%
\ assms\ Concepts{\isacharunderscore}{\kern0pt}are{\isacharunderscore}{\kern0pt}annotations\ Nonduality\ \isakeywordONE{by}\isamarkupfalse%
\ blast%
\endisatagproof
{\isafoldproof}%
%
\isadelimproof
%
\endisadelimproof
%
\isadelimdocument
%
\endisadelimdocument
%
\isatagdocument
%
\isamarkupsection{Quantities for Information and Time%
}
\isamarkuptrue%
%
\endisatagdocument
{\isafolddocument}%
%
\isadelimdocument
%
\endisadelimdocument
\isakeywordONE{typedecl}\isamarkupfalse%
\ Q%
\isadelimdocument
%
\endisadelimdocument
%
\isatagdocument
%
\isamarkupsection{Information Layer (Abstract Nonnegativity)%
}
\isamarkuptrue%
%
\endisatagdocument
{\isafolddocument}%
%
\isadelimdocument
%
\endisadelimdocument
\isakeywordONE{consts}\isamarkupfalse%
\isanewline
\ \ Info\ \ \ {\isacharcolon}{\kern0pt}{\isacharcolon}{\kern0pt}\ {\isachardoublequoteopen}E\ {\isasymRightarrow}\ Q{\isachardoublequoteclose}\isanewline
\ \ Nonneg\ {\isacharcolon}{\kern0pt}{\isacharcolon}{\kern0pt}\ {\isachardoublequoteopen}Q\ {\isasymRightarrow}\ bool{\isachardoublequoteclose}\isanewline
\isanewline
\isakeywordONE{axiomatization}\isamarkupfalse%
\ \isakeywordTWO{where}\isanewline
\ \ Info{\isacharunderscore}{\kern0pt}nonneg{\isacharcolon}{\kern0pt}\ {\isachardoublequoteopen}{\isasymforall}x{\isachardot}{\kern0pt}\ Phenomenon\ x\ {\isasymlongrightarrow}\ Nonneg\ {\isacharparenleft}{\kern0pt}Info\ x{\isacharparenright}{\kern0pt}{\isachardoublequoteclose}\isanewline
\isanewline
\isakeywordONE{lemma}\isamarkupfalse%
\ Info{\isacharunderscore}{\kern0pt}nonreifying{\isacharcolon}{\kern0pt}\isanewline
\ \ \isakeywordTWO{assumes}\ {\isachardoublequoteopen}Phenomenon\ x{\isachardoublequoteclose}\ \isakeywordTWO{shows}\ {\isachardoublequoteopen}Inseparable\ x\ {\isasymOmega}{\isachardoublequoteclose}\isanewline
%
\isadelimproof
\ \ %
\endisadelimproof
%
\isatagproof
\isakeywordONE{using}\isamarkupfalse%
\ assms\ Nonduality\ \isakeywordONE{by}\isamarkupfalse%
\ blast%
\endisatagproof
{\isafoldproof}%
%
\isadelimproof
%
\endisadelimproof
%
\isadelimdocument
%
\endisadelimdocument
%
\isatagdocument
%
\isamarkupsection{Emergent Time (Abstract Strict Order on Q)%
}
\isamarkuptrue%
%
\endisatagdocument
{\isafolddocument}%
%
\isadelimdocument
%
\endisadelimdocument
\isakeywordONE{consts}\isamarkupfalse%
\isanewline
\ \ T\ \ {\isacharcolon}{\kern0pt}{\isacharcolon}{\kern0pt}\ {\isachardoublequoteopen}E\ {\isasymRightarrow}\ Q{\isachardoublequoteclose}\ \ \ \ \ \ \ \ \ \ \ \isanewline
\ \ LT\ {\isacharcolon}{\kern0pt}{\isacharcolon}{\kern0pt}\ {\isachardoublequoteopen}Q\ {\isasymRightarrow}\ Q\ {\isasymRightarrow}\ bool{\isachardoublequoteclose}\ \ \ \isanewline
\isanewline
\isakeywordONE{axiomatization}\isamarkupfalse%
\ \isakeywordTWO{where}\isanewline
\ \ LT{\isacharunderscore}{\kern0pt}irrefl{\isacharcolon}{\kern0pt}\ \ \ \ \ {\isachardoublequoteopen}{\isasymforall}q{\isachardot}{\kern0pt}\ {\isasymnot}\ LT\ q\ q{\isachardoublequoteclose}\ \isakeywordTWO{and}\isanewline
\ \ LT{\isacharunderscore}{\kern0pt}trans{\isacharcolon}{\kern0pt}\ \ \ \ \ \ {\isachardoublequoteopen}{\isasymforall}a\ b\ c{\isachardot}{\kern0pt}\ LT\ a\ b\ {\isasymand}\ LT\ b\ c\ {\isasymlongrightarrow}\ LT\ a\ c{\isachardoublequoteclose}\ \isakeywordTWO{and}\isanewline
\ \ Time{\isacharunderscore}{\kern0pt}monotone{\isacharcolon}{\kern0pt}\ {\isachardoublequoteopen}{\isasymforall}x\ y{\isachardot}{\kern0pt}\ CausallyPrecedes\ x\ y\ {\isasymlongrightarrow}\ LT\ {\isacharparenleft}{\kern0pt}T\ x{\isacharparenright}{\kern0pt}\ {\isacharparenleft}{\kern0pt}T\ y{\isacharparenright}{\kern0pt}{\isachardoublequoteclose}\isanewline
\isanewline
\isakeywordONE{lemma}\isamarkupfalse%
\ Time{\isacharunderscore}{\kern0pt}emergent{\isacharunderscore}{\kern0pt}NotTwo{\isacharcolon}{\kern0pt}\isanewline
\ \ \isakeywordTWO{assumes}\ {\isachardoublequoteopen}Phenomenon\ x{\isachardoublequoteclose}\ \isakeywordTWO{shows}\ {\isachardoublequoteopen}Inseparable\ x\ {\isasymOmega}{\isachardoublequoteclose}\isanewline
%
\isadelimproof
\ \ %
\endisadelimproof
%
\isatagproof
\isakeywordONE{using}\isamarkupfalse%
\ assms\ Nonduality\ \isakeywordONE{by}\isamarkupfalse%
\ blast%
\endisatagproof
{\isafoldproof}%
%
\isadelimproof
%
\endisadelimproof
%
\isadelimdocument
%
\endisadelimdocument
%
\isatagdocument
%
\isamarkupsection{Two-Levels Coherence%
}
\isamarkuptrue%
%
\endisatagdocument
{\isafolddocument}%
%
\isadelimdocument
%
\endisadelimdocument
\isakeywordONE{consts}\isamarkupfalse%
\ Coherent\ {\isacharcolon}{\kern0pt}{\isacharcolon}{\kern0pt}\ {\isachardoublequoteopen}E\ {\isasymRightarrow}\ bool{\isachardoublequoteclose}\isanewline
\isanewline
\isakeywordONE{axiomatization}\isamarkupfalse%
\ \isakeywordTWO{where}\isanewline
\ \ Conventional{\isacharunderscore}{\kern0pt}is{\isacharunderscore}{\kern0pt}model{\isacharunderscore}{\kern0pt}relative{\isacharcolon}{\kern0pt}\ {\isachardoublequoteopen}{\isasymforall}x{\isachardot}{\kern0pt}\ ValidConv\ x\ {\isasymlongrightarrow}\ Phenomenon\ x{\isachardoublequoteclose}\ \isakeywordTWO{and}\isanewline
\ \ Ultimate{\isacharunderscore}{\kern0pt}coherence{\isacharcolon}{\kern0pt}\ \ \ \ \ \ \ \ \ \ \ \ \ {\isachardoublequoteopen}Coherent\ {\isasymOmega}{\isachardoublequoteclose}%
\isadelimdocument
%
\endisadelimdocument
%
\isatagdocument
%
\isamarkupsection{Notation and Robustness%
}
\isamarkuptrue%
%
\endisatagdocument
{\isafolddocument}%
%
\isadelimdocument
%
\endisadelimdocument
\isakeywordONE{definition}\isamarkupfalse%
\ NotTwo\ {\isacharcolon}{\kern0pt}{\isacharcolon}{\kern0pt}\ {\isachardoublequoteopen}E\ {\isasymRightarrow}\ E\ {\isasymRightarrow}\ bool{\isachardoublequoteclose}\isanewline
\ \ \isakeywordTWO{where}\ {\isachardoublequoteopen}NotTwo\ x\ y\ {\isasymlongleftrightarrow}\ Inseparable\ x\ y{\isachardoublequoteclose}\isanewline
\isanewline
\isakeywordONE{lemma}\isamarkupfalse%
\ Phenomenon{\isacharunderscore}{\kern0pt}NotTwo{\isacharunderscore}{\kern0pt}Base{\isacharcolon}{\kern0pt}\ {\isachardoublequoteopen}Phenomenon\ p\ {\isasymLongrightarrow}\ NotTwo\ p\ {\isasymOmega}{\isachardoublequoteclose}\isanewline
%
\isadelimproof
\ \ %
\endisadelimproof
%
\isatagproof
\isakeywordONE{using}\isamarkupfalse%
\ Nonduality\ NotTwo{\isacharunderscore}{\kern0pt}def\ \isakeywordONE{by}\isamarkupfalse%
\ blast%
\endisatagproof
{\isafoldproof}%
%
\isadelimproof
\isanewline
%
\endisadelimproof
\isanewline
\isakeywordONE{lemma}\isamarkupfalse%
\ Any{\isacharunderscore}{\kern0pt}presentation{\isacharunderscore}{\kern0pt}structure{\isacharunderscore}{\kern0pt}preserves{\isacharunderscore}{\kern0pt}NotTwo{\isacharcolon}{\kern0pt}\isanewline
\ \ \isakeywordTWO{assumes}\ {\isachardoublequoteopen}Phenomenon\ x{\isachardoublequoteclose}\ \isakeywordTWO{shows}\ {\isachardoublequoteopen}NotTwo\ x\ {\isasymOmega}{\isachardoublequoteclose}\isanewline
%
\isadelimproof
\ \ %
\endisadelimproof
%
\isatagproof
\isakeywordONE{using}\isamarkupfalse%
\ assms\ Nonduality\ NotTwo{\isacharunderscore}{\kern0pt}def\ \isakeywordONE{by}\isamarkupfalse%
\ blast%
\endisatagproof
{\isafoldproof}%
%
\isadelimproof
\isanewline
%
\endisadelimproof
%
\isadelimtheory
\isanewline
%
\endisadelimtheory
%
\isatagtheory
\isakeywordTWO{end}\isamarkupfalse%
%
\endisatagtheory
{\isafoldtheory}%
%
\isadelimtheory
%
\endisadelimtheory
%
\end{isabellebody}%
\endinput
%:%file=~/Documents/GitHub/AFP/isabelle/four_systems/NonDuality/NonDuality.thy%:%
%:%10=1%:%
%:%11=1%:%
%:%12=2%:%
%:%13=3%:%
%:%27=43%:%
%:%37=45%:%
%:%38=45%:%
%:%39=46%:%
%:%40=47%:%
%:%41=47%:%
%:%42=48%:%
%:%43=49%:%
%:%44=50%:%
%:%45=51%:%
%:%46=52%:%
%:%47=53%:%
%:%48=53%:%
%:%49=54%:%
%:%50=55%:%
%:%51=56%:%
%:%52=57%:%
%:%53=58%:%
%:%54=59%:%
%:%55=60%:%
%:%56=60%:%
%:%59=61%:%
%:%63=61%:%
%:%64=61%:%
%:%65=61%:%
%:%70=61%:%
%:%73=62%:%
%:%74=63%:%
%:%75=63%:%
%:%76=64%:%
%:%77=65%:%
%:%78=66%:%
%:%79=66%:%
%:%82=67%:%
%:%86=67%:%
%:%87=67%:%
%:%88=67%:%
%:%89=67%:%
%:%94=67%:%
%:%97=68%:%
%:%98=69%:%
%:%99=69%:%
%:%102=70%:%
%:%106=70%:%
%:%107=70%:%
%:%108=70%:%
%:%113=70%:%
%:%116=71%:%
%:%117=72%:%
%:%118=72%:%
%:%120=72%:%
%:%124=72%:%
%:%125=72%:%
%:%139=75%:%
%:%149=77%:%
%:%150=77%:%
%:%151=78%:%
%:%158=79%:%
%:%159=79%:%
%:%160=80%:%
%:%161=80%:%
%:%162=80%:%
%:%163=81%:%
%:%164=81%:%
%:%165=81%:%
%:%166=81%:%
%:%167=82%:%
%:%168=82%:%
%:%169=83%:%
%:%170=83%:%
%:%171=83%:%
%:%172=84%:%
%:%173=84%:%
%:%174=85%:%
%:%175=85%:%
%:%176=85%:%
%:%177=86%:%
%:%192=88%:%
%:%202=90%:%
%:%203=90%:%
%:%204=91%:%
%:%205=92%:%
%:%206=92%:%
%:%207=93%:%
%:%208=94%:%
%:%209=95%:%
%:%210=96%:%
%:%211=97%:%
%:%212=97%:%
%:%213=98%:%
%:%216=99%:%
%:%220=99%:%
%:%221=99%:%
%:%222=99%:%
%:%227=99%:%
%:%230=100%:%
%:%231=101%:%
%:%232=101%:%
%:%233=102%:%
%:%236=103%:%
%:%240=103%:%
%:%241=103%:%
%:%242=103%:%
%:%256=105%:%
%:%266=107%:%
%:%267=107%:%
%:%268=108%:%
%:%269=108%:%
%:%270=109%:%
%:%271=110%:%
%:%272=110%:%
%:%273=111%:%
%:%274=112%:%
%:%275=113%:%
%:%276=114%:%
%:%277=114%:%
%:%278=115%:%
%:%279=116%:%
%:%280=117%:%
%:%281=118%:%
%:%282=119%:%
%:%283=120%:%
%:%284=120%:%
%:%285=121%:%
%:%286=122%:%
%:%293=123%:%
%:%294=123%:%
%:%295=124%:%
%:%296=124%:%
%:%297=124%:%
%:%298=124%:%
%:%299=125%:%
%:%300=125%:%
%:%301=125%:%
%:%302=125%:%
%:%303=126%:%
%:%304=126%:%
%:%305=126%:%
%:%306=126%:%
%:%307=127%:%
%:%322=129%:%
%:%332=131%:%
%:%333=131%:%
%:%334=132%:%
%:%335=133%:%
%:%336=133%:%
%:%337=134%:%
%:%344=136%:%
%:%354=138%:%
%:%355=138%:%
%:%356=139%:%
%:%357=140%:%
%:%358=140%:%
%:%359=141%:%
%:%360=142%:%
%:%361=143%:%
%:%368=145%:%
%:%378=147%:%
%:%379=147%:%
%:%380=148%:%
%:%381=148%:%
%:%382=149%:%
%:%383=149%:%
%:%384=150%:%
%:%385=151%:%
%:%386=151%:%
%:%387=152%:%
%:%388=153%:%
%:%389=154%:%
%:%390=155%:%
%:%397=157%:%
%:%407=159%:%
%:%408=159%:%
%:%409=160%:%
%:%410=160%:%
%:%411=161%:%
%:%412=162%:%
%:%413=162%:%
%:%414=163%:%
%:%415=164%:%
%:%416=165%:%
%:%417=166%:%
%:%418=166%:%
%:%419=167%:%
%:%420=168%:%
%:%423=169%:%
%:%427=169%:%
%:%428=169%:%
%:%429=170%:%
%:%430=170%:%
%:%444=173%:%
%:%454=175%:%
%:%455=175%:%
%:%456=176%:%
%:%457=176%:%
%:%458=177%:%
%:%459=178%:%
%:%460=178%:%
%:%461=179%:%
%:%462=180%:%
%:%463=181%:%
%:%464=182%:%
%:%465=182%:%
%:%466=183%:%
%:%469=184%:%
%:%473=184%:%
%:%474=184%:%
%:%475=184%:%
%:%489=186%:%
%:%499=188%:%
%:%500=188%:%
%:%507=190%:%
%:%517=192%:%
%:%518=192%:%
%:%519=193%:%
%:%520=194%:%
%:%521=195%:%
%:%522=196%:%
%:%523=196%:%
%:%524=197%:%
%:%525=198%:%
%:%526=199%:%
%:%527=199%:%
%:%528=200%:%
%:%531=201%:%
%:%535=201%:%
%:%536=201%:%
%:%537=201%:%
%:%551=203%:%
%:%561=205%:%
%:%562=205%:%
%:%563=206%:%
%:%564=207%:%
%:%565=208%:%
%:%566=209%:%
%:%567=209%:%
%:%568=210%:%
%:%569=211%:%
%:%570=212%:%
%:%571=213%:%
%:%572=214%:%
%:%573=214%:%
%:%574=215%:%
%:%577=216%:%
%:%581=216%:%
%:%582=216%:%
%:%583=216%:%
%:%597=218%:%
%:%607=220%:%
%:%608=220%:%
%:%609=221%:%
%:%610=222%:%
%:%611=222%:%
%:%612=223%:%
%:%613=224%:%
%:%620=226%:%
%:%630=228%:%
%:%631=228%:%
%:%632=229%:%
%:%633=230%:%
%:%634=231%:%
%:%635=231%:%
%:%638=232%:%
%:%642=232%:%
%:%643=232%:%
%:%644=232%:%
%:%649=232%:%
%:%652=233%:%
%:%653=234%:%
%:%654=234%:%
%:%655=235%:%
%:%658=236%:%
%:%662=236%:%
%:%663=236%:%
%:%664=236%:%
%:%669=236%:%
%:%674=237%:%
%:%679=238%:%



\end{document}
\endinput
%:%file=~/Documents/GitHub/AFP/isabelle/four_systems/NonDuality/document/root.tex%:%

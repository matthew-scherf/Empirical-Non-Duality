\documentclass[11pt,a4paper]{article}

% Isabelle document support
\usepackage{isabelle,isabellesym}

% Text + monospace quality
\usepackage[T1]{fontenc}
\usepackage[utf8]{inputenc}
\usepackage{inconsolata}

% Hyperlinks
\usepackage{hyperref}

% Safe code listings for inline examples in the prose
\usepackage{listings}
\lstset{
  basicstyle=\ttfamily\small,
  columns=fullflexible,
  keepspaces=true,
  showstringspaces=false,
  upquote=true,
  frame=single,
  breaklines=true
}

\title{Non-Duality: A Minimal Isabelle/HOL Formalization}
\author{Matthew Scherf}
\date{\today}

\begin{document}

\maketitle

\begin{abstract}
This entry presents a concise formalization of an empirically oriented
non-dual ontology in Isabelle/HOL. The development isolates a minimal core
comprising a unique ground (denoted \(\Omega\)), phenomena, an inseparability
relation, and a symmetry action on appearances. Within Higher-Order Logic, we
state conservative axioms and derive representative consequences such as the
dependence of phenomena on the ground, the non-duplication of ground, and the
stability of inseparability under symmetry. The theory is designed to be small,
transparent, and reusable, providing a neutral spine that can be compared with
neighboring non-dual formalisations (e.g.\ Daoism, Dzogchen) in the same
technical framework.
\end{abstract}

\tableofcontents
\bigskip

\section{Introduction}

Non-dual traditions often assert that appearances are not ultimately separate
from their ground, and that the ground is unique and inexpressible by the
categories that classify appearances. This formalization extracts a neutral,
logically explicit fragment of such views. The aim is expository and pragmatic:
to state a modest family of axioms in HOL, study their immediate consequences,
and make the assumptions and inferential pathways mechanically auditable.

\section{Entry Overview}

The session consists of a single theory \texttt{NonDuality} importing
\texttt{Main}. At a high level it introduces:
\begin{itemize}
  \item a background type of entities;
  \item a distinguished constant \(\Omega\) intended as the unique ground;
  \item predicates capturing appearances (\texttt{Phenomenon}) and
        inseparability (\texttt{Inseparable});
  \item a (group-like) action \texttt{act} for symmetry on appearances;
  \item axioms expressing uniqueness of ground, dependence of phenomena on the
        ground, and stability properties such as closure and preservation under
        action.
\end{itemize}
The development proves elementary consequences and keeps proofs short and
readable. It avoids interactive tool invocations in batch builds.

\section{Logical Design}

\subsection{Core Symbols (Sketch)}

We work in classical HOL with a small signature. A typical abridged header:
\begin{lstlisting}
typedecl entity

consts
  Omega        :: entity
  Phenomenon   :: "entity ⇒ bool"
  Inseparable  :: "entity ⇒ entity ⇒ bool"
  act          :: "('g) ⇒ entity ⇒ entity"   (* symmetry action on appearances *)
\end{lstlisting}

Nothing in the entry depends on a particular representation of the carrier for
symmetries; the action is treated abstractly.

\subsection{Axiom Sketch}

The axioms are minimal and phenomenology-friendly. Representative principles:
\begin{itemize}
  \item \emph{Unique ground:} \(\Omega\) is the ground and there is no other
        distinct ground entity.
  \item \emph{Ground--appearance relation:} every phenomenon is inseparable
        from \(\Omega\).
  \item \emph{Closure and preservation:} the symmetry action maps phenomena to
        phenomena and preserves inseparability with \(\Omega\).
\end{itemize}
The theory presents these as plain HOL formulas without requiring non-classical
logic or additional meta-theory.

\section{Illustrative Results}

\paragraph{Dependence of Phenomena on the Ground.}
Every appearance is inseparable from \(\Omega\):
\begin{lstlisting}
lemma phenomenon_inseparable_from_Omega:
  assumes "Phenomenon x"
  shows   "Inseparable x Ω"
\end{lstlisting}

\paragraph{Symmetry Preserves Non-Two.}
Inseparability is stable under the action:
\begin{lstlisting}
lemma symmetry_preserves_inseparability:
  assumes "Phenomenon x"
  shows   "Inseparable (act g x) Ω"
\end{lstlisting}

\paragraph{No Duplication of Ground.}
There is no second ground distinct from \(\Omega\):
\begin{lstlisting}
lemma ground_is_unique:
  assumes "Inseparable Ω y" "Inseparable y Ω"
  shows   "y = Ω"
\end{lstlisting}

These statements serve as precise surrogates for familiar non-dual claims.
They are proven under the minimal axioms and can be strengthened or varied by
altering the axiomatic core.

\section{Methods and Sanity Checks}

Proofs prefer \texttt{auto}, \texttt{blast}, and \texttt{metis} and avoid
non-deterministic or long-running tools in batch mode. During development,
finite-scope model exploration (via Nitpick) and proof search (via Sledgehammer)
were used interactively to guide axiom tuning and lemma selection; the final
submitted theory contains only stable proofs.%
\footnote{Typical interactive settings: small cardinalities and short timeouts
for model exploration; proofs subsequently recorded with \texttt{metis} or
\texttt{blast}.}

\section{Relation to Prior Work}

This entry complements other small, reusable non-dual formalisations (e.g.\
Daoism, Dzogchen) by offering a secular, ontology-first presentation focused on
a unique ground \(\Omega\) and an inseparability relation. Placing these within
the same HOL substrate supports side-by-side comparison and future unification
work.

\section{Limitations and Future Work}

The axioms intentionally abstract away from interpretive nuance. Natural
extensions include:
\begin{itemize}
  \item enriching \texttt{Inseparable} with modal or temporal structure;
  \item constraining \texttt{act} via group laws and equivariance properties;
  \item relating the core to epistemic surrogates of direct knowing;
  \item comparing conservativity across neighboring entries within a shared
        interface theory.
\end{itemize}

\section{How to Build}

From the session directory:
\begin{lstlisting}
isabelle build -c -o document=pdf NonDuality
\end{lstlisting}
To emit the PDF alongside sources:
\begin{lstlisting}
isabelle build -c -D . -o document=pdf -o document_output=public NonDuality
\end{lstlisting}

\bigskip

% ---------------------------------------------------------------------------

\input{session}

\end{document}
\endinput
%:%file=~/Documents/GitHub/AFP/isabelle/four_systems/NonDuality/document/root.tex%:%
